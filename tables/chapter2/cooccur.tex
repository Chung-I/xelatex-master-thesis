以上四句話的共現矩陣為:
\begin{table}[ht]
    \center
    \begin{tabular}{|c|c|c|c|c|c|c|c|c|} \hline
    %\begin{tabularx}{\textwidth}{|*{9}{X|}} \hline
           & 李宏毅 & 李仲翊 & 幾 & 班 & 三 & 五 & 年 & 二十  \\ \hline
     李宏毅 & 0 & 0 & 1 & 2 & 0 & 1 & 1 & 1 \\ \hline
     李仲翊 & 0 & 0 & 1 & 2 & 1 & 1 & 1 & 0 \\ \hline
        幾 & 1 & 1 & 0 & 2 & 0 & 0 & 0 & 0 \\ \hline
        班 & 2 & 2 & 2 & 0 & 1 & 2 & 2 & 1 \\ \hline
        三 & 0 & 1 & 0 & 1 & 0 & 1 & 1 & 0 \\ \hline
        五 & 1 & 1 & 0 & 2 & 1 & 0 & 2 & 1 \\ \hline
        年 & 1 & 1 & 0 & 2 & 1 & 2 & 0 & 1 \\ \hline
      二十 & 1 & 0 & 0 & 1 & 0 & 1 & 1 & 0 \\ \hline
    \end{tabular}
    %\end{tabularx}
    \caption{共現矩陣的示例。}
    \label{tab:cooccur}
\end{table}

表\ref{tab:cooccur}中,每個字的分佈表示即爲該欄/列向量(在對稱的共現矩陣中欄向量與列向量相同)。
